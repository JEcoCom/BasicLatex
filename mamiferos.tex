\documentclass[a4paper,16,onecolumn]{article}
\usepackage[utf8]{inputenc}
\usepackage[portuguese]{babel}
\usepackage{graphicx}
\author{João Caetano}
\title{Pesquisa para a realização do projeto mamíferos}
\begin{document}
\maketitle
\tableofcontents
\pagebreak
\section{Fundamentação Básica}
Qual é a fundamentação básica para a criação de um conhecimento relacionado ao manejo de fauna??
\begin{itemize}
\item Ocupação dos ambientes naturais em que os animais originalmente viviam.
\item Adaptação dos animais para se adaptar aos novos recursos disponíveis ao ambiente.
\section{Idéias Importantes}
\begin{itemize}
\item Considerar informações de outros grupos taxonomicos pode ser um ponto importante para se avaliar a questão dos cachorros do mato uma vez que esses animais não tem estudos tão aprofundados como os realizados em países que investem mais em pesquisa.
\item Excluder

\subsection{Pontos importantes a serem considerados}
\begin{itemize}
\item O documento da FAA é uma importante referencia a ser considerada para montar o projeto
\item Conceito de espécies com tolerancia zero.
\item Pesticidas para grama
\item Revisão programada dos métodos a cada 12 meses 
\item Plano de treinamento para equipes do aeroporto
\item Realizar uma pesquisa sobre a legislação pertinente ao manejo de fauna em aeroportos.
\item Um problema ao se tratar desse tema é jusntamente a baixa quantidade de informações relacionada a parte de aplicação de manejo de fauna e a criação de indicadore que ajundem que possam orientar as entidades aeroportuárias.
\item Um ponto interessante a ser pesquisado posteriormente são as metodologias para controle de cervideos apresentadas no capítulo 1(\cite{Adams2016}). O exemplo apresenta uma compilação dos artigos que foram publicados sobre o tema, os quais podem ter uma utilidade para a aplicação no controle de fauna em aeroportos.

\end{itemize}

\end{itemize}
\section{Fundamentaçãoo do projeto}

\ A presença de mamíferos no Aeroporto Internacional de Brasília pode ser um risco em potencial para a operação de aeronaves em sua localidae. Com o objetivo de levantar informações para avaliar o nível de risco desse fator para a operação de aeronaves na localidade e poder se tomar atitudes para mitigar os riscos relacionados a esse tema no aerodromo esse documento está sendo desenvolvido com o objetivo de mitigar os riscos da presença de mamíferos no ambiente do aerodromo.

\ O risco da presença de mamíferos na localidade do aerodramo tem sido evidenciado tanto pela constante captura de animais detro das áreas operacionais do aerodromo como na ocorrencia de colisões com esse tipo de fauna em suas imediações. O fato de o aeroporto estar localizado em um ambiente com características tanto Urbanas como de Vegetação natural traz oportunidade de colisão tanto com fauna doméstica como com animais silvestres. Algus exemplos de animais que já foram registrados utilizando as áreas operacionais do aeródromo são os cachorros domésticos, gatos domésticos, capivaras, Cachorro-do-Mato(\emph{Cerdocyon thous}), raposa-do-campo(\emph{Lycalopex velutus}), Lobo-Guará(\emph{Chrysocyon brachyurus}), tamanduá-bandeira(\emph{Myrmecophaga tridactyla}), tamanduá-Mirim(\emph{Tamandua tetradactyla}), Cutia(\emph{Dasyprocta azarae}), Furão(\emph{Galictis vittata}), Saruê(\emph{Didelphis albiventris}). Outras espécie que apresenta um grande potencial para a ocorrencia de colisões nas imediações do sítio aeroportuário mas que nunca foi registrada dentro das áreas operacionais a Anta(\emph{Tapirus terrestris}).

\ A principal vulnerabilidade do aeroporto atualmente em relação à presença de mamíferos de grande porte é a presença de alimento e abrigo nos enclaves de cerrado que estão localisados dentro da área operacional do aeródromo. O segundo grande problema é a falta de barreira físca eficiente que impessa o transito dos animais, uma vez que as cercas utilizadas no ambiente do aeroporto atualmente apresentam diversas vulnerabilidades, permitindo assim o acesso desses animais ao ambiente operacional.

\ Ser capaz de monitorar e manejar esses animais que estão presentes no ambiente é de vital importância para as operações no aerodromo. Dessa forma a aplicação de metodologia de manejo da vida selvagem se torna indispensável para mitigar e prever os riscos de colisões com esse tipo de fauna.

\section{Comportamento}
\subsection{Utilização do Território}

Realizo nesse ponto uma inferencia de comportamento relacionando os cachorros do mato com coiotes da américa do norte, os quais são animais que possuem estudos muito aprofundados de comportamento. 

A utilização de território pelos coiotes tem aspéctos que podem ser de direto interesse para os aeroportos, uma vez que algumas diretrizes podem ser utilizadas para se trabalhar nesse tema. Um aspécto interessante que trago é a reitirada constante de animais do ambiente acaba sendo um processo sem fim devido à atração de outros animais para o território desprotegido \cite{Lehner1976} . Dessa forma A reitirada de indivíduos do ambiente resulta em um espaço vazio que acaba atraindo outros individuos para o mesmo ambiente.
\subsection{Fundamentos da Ecologia Urbana}


\section{Metodologias de Monitoramento}
\subsection{Recomendações da FAA para realizar o Manejo de Mamíferos \cite{FAA2018}}
\begin{itemize}
\item Realização de monitoramentos noturnos utilizando holofotes(Frequência: Uma vez a cada 4 meses)
\item Monitoramento de roedores para avaliar a correlação da presença dos mesmos com relaçãoo à presença de predadores.
\item Spot Maping
\item Monitoramento utilizando camera trapps(Não citado no documento)
\item Mapeamento de atrativos

\end{itemize}

\section{Metodologias a Serem Empregadas no monitoramento}

\subsection{Armadilhagem}

\ A armadilhagem é a metodologia mais básica e mais importante que vem sendo desenvolvida no Aeroporto de Brasília para trabalhar com os mamíferos que adentram as áreas operacionais do aerodromo. A utilização dessa metodologia é de vital importância para retirar animais que adentram a área operacional e em conjunto com os métodos de busca de rastros e camera traps auxiliam na detecção mamíferos que estejam presentes na área operacional.

\ Um problema do atual método de armadilhagem é que as informações coletada durante esse processo não traz informações úteis para analise e padronização dos conjuntos de dados uma vez que não há uma padronização da metodologia de amostragem. A falta de padronização limita uma recuperação dos dados rapidamente, além de impedir que se ter uma medida precisa do esforço de captura.

\ A utilização de um método para lidar com essa situação se mostra-se de vital importância, de forma a ter um banco de dados que possa embasar as ações da equipe de manejo de fauna a longo prazo, além de ter uma forma rápida e objetiva de apresentar resultados para os contratantes e para a comunidade aeroportuária.

\ Dessa fora a atual proposta para a armadilhagem é a disposição de armadilhas fixas constantemente iscadas dentro do ambiente aeroportuário assim como estava sendo feito anteriormente, mas dessa vez as mesmas serão dispostas de maneira mais uniforme de forma a ser possível se detectar as localidades em que os animais estão localizados. De forma a conseguir cobrir todas as localidade da área operacional será necessário utilizar 18 armadilhas fixas com 10 armadilhas na pista 11L29R e 8 armadilhas na pista 11R29L. A primeira pista recebe mais armadilhas com o objetivo de cobrir as áreas de enclaves de cerrado que estão localizadas dentro da área operacional nessa pista.

\ As armadilhas serão iscadas e acompanhadas semanalmente com o objetivo de verificar as capturas e os animais capturados serão devidamente direcionados os objetivos definidos de acordo com o plano de manejo estabelecido para o Aeroporto de Brasília.

\begin{figure}[!h]
\includegraphics[width=\linewidth]{1.jpeg}
\caption{Localização das armadilhas na pista 11L29R.}
\label{fig:1}
\includegraphics[width=\linewidth]{2.jpeg}
\caption{Localização das armadilhas na pista 11R29L.}
\label{fig:2}
\end{figure}
\clearpage

\subsection{Vistoria de Rastros}
\subsection{Camera Trapps}
\subsection{Registros Aleatórios}

\section{Artigos interessantes para serem pesquisados}
\begin{itemize}
\item Some observations of coyote predation in Yellowstone National Park. Robinson,1952
\item God's dog. Ryden,1975

\end{itemize}

\section{Trabalhos interessantes a serem observados}
\begin{itemize}
\item Livro Urban Wildlife manegement
\subitem Capítulo 2 Principal Components of Urban Wildlife Management.
\subitem Capitulo 11 Urban Mammals.

\end{itemize}

\section{Bibliografia}
\begin{thebibliography}{10}

\bibitem{Lehner1976}
 Lehner.
\textit{Coyote behavior: Implications for manegement}
(1976).

\bibitem{FAA2018}
FAA.
\textit{Protocol for the Conduct and Review of Wildlife Hazard Sites Visits, Wildlife Hazard Assessments, and Wildlife Hazard Management Plans.} 
(2018), Advisory Circular.

\bibitem{Adams2016}
Adams, C. E
\textit{Urban Wildlife Manegement}
(2016),Book

\end{thebibliography}

\end{document}

	